% [S:1]
% {org}
% Las lenguas habladas por un gran grupo de personas son menos vulnerables
% al peligro de desaparecer que otras.
% {/org}
% {unl}
% mod(tongue(icl>concrete_thing,pof>body):04.@def.@pl, spoken(icl>adj):0C)
% pos(group(icl>group,mod>thing):0X.@indef, person(icl>person):16.@pl)
% mod(group(icl>group,mod>thing):0X.@indef, large(icl>adj):0S)
% obj(vulnerable(icl>adj):06.@entry,
% danger(icl>abstract_thing,obj>uw):0L.@def)
% man(vulnerable(icl>adj):06.@entry, less(icl>how):00)
% obj(danger(icl>abstract_thing,obj>uw):0L.@def,
% disappear(icl>occur,equ>vanish,obj>thing,plf>uw):0W)
% agt(spoken(icl>adj):0C, group(icl>group,mod>thing):0X.@indef)
% aoj(vulnerable(icl>adj):06.@entry,
% tongue(icl>concrete_thing,pof>body):04.@def.@pl)
% bas(less(icl>how):00, other(icl>adj,equ> additional):00.@pl)
% {/unl}
% [/S]

\documentclass{article}
\usepackage[margin=0.2cm]{geometry}

\begin{document}

    Listado de Sustantivos:
    \begin{itemize}
        \item Alfombra
        \begin{verbatim}
            carpet(icl>floor_cover).@sg
        \end{verbatim}
        \item Cama
        \begin{verbatim}
            bed(icl>bedroom_furniture).@sg
        \end{verbatim}
        \item Casa
        \begin{verbatim}
            house(icl>home).@sg
        \end{verbatim}
        \item Color*
        \begin{verbatim}
            colour(icl>visual_property, aoj>concrete_thing, obj>colour).@sg
            El libro[aoj] es de color rojo[obj] 
        \end{verbatim}
        \item Cosas
        \begin{verbatim}
            thing(icl>physical_entity).@pl
        \end{verbatim}
        \item Cuadros
        \begin{verbatim}
            painting(icl>graphic_art).@pl
        \end{verbatim}
        \item Ejercicios*
        \begin{verbatim}
            exercise(icl>lesson, aoj>abstract_thing).@pl
            Utilizo mi libro de ejercicios (de español [aoj]) para aprender mejor el español.
        \end{verbatim}
        \item Escritorio
        \begin{verbatim}
            desk(icl>table).@sg
        \end{verbatim}
        \item Espacio
        \begin{verbatim}
            space(icl>location).@sg
        \end{verbatim}
        \item Español*
        \begin{verbatim}
            spanish(iof>romance_language).@sg
        \end{verbatim}
        \item Esquina
        \begin{verbatim}
            corner(icl>area, com>interior).@sg
        \end{verbatim}
        \item Estante
        \begin{verbatim}
            shelf(icl>support).@sg
        \end{verbatim}
        \item Fondo*
        \begin{verbatim}
            background(icl>view, aoj>concrete_thing).@sg
            Se ve al fondo (de la habitación)[aoj] la ventana.
        \end{verbatim}
        \item Habitación*
        \begin{verbatim}
            bedroom(icl>room, pof>house).@sg
        \end{verbatim}
        \item Hogar
        \begin{verbatim}
            home(icl>residence).@sg
        \end{verbatim}
        \item Infancia
        \begin{verbatim}
            childhood(icl>time_of_life).@sg
        \end{verbatim}
        \item Juguetes
        \begin{verbatim}
            toy(icl>artifact).@pl
        \end{verbatim}
        \item Libro, Libros**
        \begin{verbatim}
            book(icl>publication, aoj>abstract_thing).@pl
            book(icl>publication, aoj>abstract_thing).@sg
            Leo un libro sobre el español (aoj)
            book(icl>publication)
        \end{verbatim}
        \item Lugar, Lugares*
        \begin{verbatim}
            place(icl>point).@sg
        \end{verbatim}
        \item Luz
        \begin{verbatim}
            light(icl>ilumination).@sg
        \end{verbatim}
        \item Lámpara
        \begin{verbatim}
            lamp(icl>source_of_illumination).@sg
        \end{verbatim}
        \item Música
        \begin{verbatim}
            music(icl>sound).@sg
        \end{verbatim}
        \item Novelas
        \begin{verbatim}
            novel(icl>book).@sg
        \end{verbatim}
        \item Objetos
        \begin{verbatim}
            object(icl>physical_entity).@pl
        \end{verbatim}
        \item Paredes*
        \begin{verbatim}
            wall(icl>divider, pof>room).@pl
        \end{verbatim}
        \item Parte*
        \begin{verbatim}
            part(icl>relation, aoj>abstract_thing).@sg
            Cierta parte de mi tiempo (aoj)
        \end{verbatim}
        \item Puerta
        \begin{verbatim}
            door(icl>movable_barrier).@sg
        \end{verbatim}
        \item Silla
        \begin{verbatim}
            chair(icl>seat).@sg
        \end{verbatim}
        \item Techo*
        \begin{verbatim}
            roof(icl>protective_covering, pof>house)
        \end{verbatim}
        \item Tiempo
        \begin{verbatim}
            time(icl>time_period).@sg
        \end{verbatim}
        \item Tono*
        \begin{verbatim}
            tone(icl>colour, aoj>concrete_thing, obj>abstract_thing).@sg
            El libro[aoj] es de un tono oscuro[obj] 
        \end{verbatim}
        \item Ventana
        \begin{verbatim}
            window(icl>frame).@sg
        \end{verbatim}
    \end{itemize}
    Listado de Verbos:
    \begin{itemize}
        \item almaceno*
        \begin{verbatim}
            store(icl>do, agt>animal, obj>concrete_thing, plc>thing).@sg.@present
            Yo[agt] almaceno cocacola[obj] en la *despensa[plc].
        \end{verbatim}
        \item aprender
        \begin{verbatim}
            learn(icl>do, agt>animal, obj>abstract_thing)
            Yo[agt] quiero aprender UNL[obj].
        \end{verbatim}
        \item contemplar*
        \begin{verbatim}
            contemplate(icl>do, agt>animal, *obj>thing)
            Es divertido contemplar *(Alguien)[agt] el paisaje[obj].
        \end{verbatim}
        % \item deja*
        % \begin{verbatim}
        %     let(icl>do, agt>thing, *obj>thing).@sg.@present
        %     La ventana[agt] deja entrar[obj] la luz.
        % \end{verbatim}
        \item descansar* (tipo be?)
        \begin{verbatim}
            rest(icl>do, agt>animal)
            Al panda[agt] le gusta descansar.
        \end{verbatim}
        \item dibujando*
        \begin{verbatim}
            draw(icl>do, agt>human, obj>thing*).@progress
            Yo[agt] estoy dibujando un perro[obj].
        \end{verbatim}
        \item divertirme* be?
        \begin{verbatim}
            have_fun(icl>be, aoj>human).@sg
            Yo[aoj] quiero divertirme.
        \end{verbatim}
        \item deja entrar*
        \begin{verbatim}
            let_in(icl>do, agt>thing, *obj>thing, plt>thing).@sg.@present
            La ventana[agt] deja (entrar la luz)[obj] a la habitación[plt].
        \end{verbatim}
        \item es
        \begin{verbatim}
            be(icl>be, obj>uw, aoj>uw).@sg.@present
            El coche[aoj] es azul[obj].
        \end{verbatim}
        \item escuchando* (escuchar a alguiem vs sonido)
        \begin{verbatim}
            listen(icl>do, agt>animal, obj>thing).@progress
            Javier[agt] está escuchando música[obj].
        \end{verbatim}
        \item están
        \begin{verbatim}
            be(icl>be, obj>uw, aoj>uw).@pl.@present
            Los niños[aoj] están felices[obj].
        \end{verbatim}
        \item estirarme
        \begin{verbatim}
            stretch(icl>be, aoj>animal).@sg
            Yo[aoj] voy a estirarme.
        \end{verbatim}
        \item estudio
        \begin{verbatim}
            study(icl>do, agt>human, obj>thing).@sg.@present
            Yo[agt] estudio redes bayesianas[obj].
        \end{verbatim}
        \item gusta, gustan
        \begin{verbatim}
            like(icl>be, aoj>animal, obj>thing).@sg.@present
            like(icl>be, aoj>animal, obj>thing).@pl.@present
            A María[aoj] le gustan las galletas María[obj].
        \end{verbatim}
        \item he colocado
        \begin{verbatim}
            
        \end{verbatim}
        \item leer, leyendo
        \begin{verbatim}
            
        \end{verbatim}
        \item llego
        \begin{verbatim}
            
        \end{verbatim}
        \item paso
        \begin{verbatim}
            
        \end{verbatim}
        \item puedo imaginar*
        \begin{verbatim}
            
        \end{verbatim}
        \item recuerdan
        \begin{verbatim}
            
        \end{verbatim}
        \item se ve
        \begin{verbatim}
            
        \end{verbatim}
        \item siento
        \begin{verbatim}
            
        \end{verbatim}
        \item tengo, tienen
        \begin{verbatim}
            
        \end{verbatim}
        \item utilizo
        \begin{verbatim}
            
        \end{verbatim}
        \item voy
        \begin{verbatim}
            
        \end{verbatim}
    \end{itemize}
\end{document}