\documentclass{article}
\usepackage[paperwidth=297mm,paperheight=210mm, margin=0.3cm]{geometry}
\usepackage{ifthen}
\usepackage{blindtext}
\usepackage{multicol}
\usepackage{enumitem}


% Remove page number
\pagestyle{empty}

\begin{document}

\section*{\huge{\underline{Ingeniería Lingüística. \emph{Cheat sheet} de UNL}}}

\begin{multicols*}{3}

\subsection*{\underline{Atributos de las UW}}

Expresan información propia del contexto. Afecta al hablante, al enunciado, modalidad. Por ejemplo, para indicar que una UW es plural:

\begin{verbatim}
tongue(icl>abstract_thing,equ>language):09.@pl
\end{verbatim}

\begin{itemize}[itemsep=0pt, parsep=0pt, leftmargin=1em]
    \item \textbf{Número} (singular o plural - @sg,@pl)
    \item \textbf{Referencia} (definida, indefinida, genérica - @def, @indef). Ej: artículo determinado / indeterminado.
    \item \textbf{Modalidad} (grado de certidumbre - @not, @possible...)
    \item \textbf{Tiempo} (pasado, presente, futuro - @present, @past...)
    \item \textbf{Intención} del hablante (ruego, invitación, orden - @order...)
    \item \textbf{Información} adicional (@entry – nodo principal del grafo)
\end{itemize}

\subsection*{\underline{Definición de UWs}}

\begin{verbatim}
UW = Headword(restricciones)
\end{verbatim}

\subsubsection*{Headwords}

Expresan de forma lo más genérica posible el significado, para luego matizarse con restricciones. Tipos:
\begin{itemize}[itemsep=0pt, parsep=0pt, leftmargin=1em]
    \item \textbf{Palabra inglesa}. Debe estar en forma de diccionario (la más básica). Ejemplo: \emph{fish}, que debe ser matizado con \emph{living\_thing} o \emph{food} para significar respectivamente pez o pescado.
    \item \textbf{Frase inglesa}. Solamente válido cuando no se puede utilizar una sola headword para representar el significado local. Ejemplo: \emph{after\_meal\_conversation} para sobremesa. 
    \item \textbf{Palabra no inglesa}. Ejemplo: \emph{paella}.
\end{itemize}

\subsubsection*{Restricciones}

Restringen el significado general de las Headwords. Tienen funcionalidad ontológica, semántica y argumental.


\begin{itemize}[itemsep=0pt, parsep=0pt, leftmargin=1em]
    \item \underline{\textbf{$>$icl}}. Conecta una UW con su clase natural (hiperónimo: palabra cuyo significado engloba el de otros). \verb+bus(icl>vehículo)+. Algunos icl útiles:
    \begin{itemize}[itemsep=0pt, parsep=0pt, leftmargin=1em]
        \item \textbf{verb}. Forma más general de describir un verbo; evitar usar.
        \begin{itemize}[itemsep=0pt, parsep=0pt, leftmargin=1em]
            \item do. Acciones con un agente que las empieza.
            \item occur. Eventos sin un agente responsable.
            \item be. Situaciones estáticas (estados, propiedades, relaciones...)
        \end{itemize}
        \item \textbf{thing}. Forma más general de describir un sustantivo; evitar usar.
        \begin{itemize}[itemsep=0pt, parsep=0pt, leftmargin=1em]
            \item abstract\_thing (action, activity, process, quantity...)
            \item concrete\_thing (living\_thing(person, animal, plant), functional\_thing, matter, natural\_object)
            \item place.
        \end{itemize}
        \item \textbf{adj}. Forma más general de describir un adjetivo.
        \item \textbf{adv}. Forma más general de describir un adverbio.
    \end{itemize}
    \item \underline{\textbf{$>$iof}}. Conecta objetos únicos con el concepto del que son instancia. \verb+Barcelona(iof>city)+, \verb+Barcelona(iof>sport_club)+, \verb+Cervantes(iof>person)+.
    \item \textbf{$>$equ}. Indica un sinónimo de la UW. \verb+UN(equ>United_Nations)+
    \item \textbf{$>$ant}. Indica un antónimo de la UW. \verb+poor(ant>rich)+
    \item \textbf{$>$pof}. Indica que la UW es un componente de un todo. \verb+roof(pof>house)+.
    \item \textbf{$>$com}. Indica un componente importante del sentido de la UW. \verb+corner(icl>place, com>interior)+
    \item \textbf{$>$agt}. Indica el agente de una acción. Ejemplo: \verb+marry(agt>human)+ si traducimos del español, pero \verb+marry(agt>male)+ o \verb+marry(agt>female)+ desde el ruso.
    \item   
\end{itemize}




\end{multicols*}

\end{document}