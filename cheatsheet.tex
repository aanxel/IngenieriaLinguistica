\documentclass{article}
\usepackage[paperwidth=297mm,paperheight=210mm, margin=0.3cm]{geometry}
\usepackage{ifthen}
\usepackage{blindtext}
\usepackage{multicol}
\usepackage{enumitem}
\usepackage[parfill]{parskip}
\usepackage{titlesec}

% General styling
% Remove page number
\pagestyle{empty}
% No indentation
\setlength{\parindent}{0pt}
% section spacing(left, before, after)
\titlespacing*{\subsubsection}
{0pt}{6pt}{8pt}

\begin{document}

\section*{\huge{\underline{Ingeniería Lingüística. \emph{Cheat sheet} de UNL}}}

\begin{multicols*}{3}

\textbf {UW = Headword(restricciones)}

\subsubsection*{\underline{Atributos de las UW}}

Ej. plural de lengua: \verb+tongue(icl>abstract_thing):09.@pl+

\begin{itemize}[itemsep=0pt, parsep=0pt, leftmargin=1em]
    \item \textbf{Número} (singular o plural - @sg,@pl)
    \item \textbf{Referencia} (definida, indefinida, genérica - @def, @indef). Ej: artículo determinado / indeterminado.
    \item \textbf{Modalidad} (grado de certidumbre - @not, @possible...)
    \item \textbf{Tiempo} (pasado, presente, futuro - @present, @past...)
    \item \textbf{Intención} del hablante (ruego, invitación, orden - @order...)
    \item \textbf{Información} adicional (@entry – nodo principal del grafo)
\end{itemize}

\subsubsection*{\underline{Headwords}}

\begin{itemize}[itemsep=0pt, parsep=0pt, leftmargin=1em]
    \item \textbf{Palabra inglesa}. Debe estar en forma de diccionario (la más básica). Ejemplo: \emph{fish}, que debe ser matizado con \emph{living\_thing} o \emph{food} para significar respectivamente pez o pescado.
    \item \textbf{Frase inglesa}. Solamente válido cuando no se puede utilizar una sola headword para representar el significado local. Ejemplo: \emph{after\_meal\_conversation} para sobremesa. 
    \item \textbf{Palabra no inglesa}. Ejemplo: \emph{paella}.
\end{itemize}

\subsubsection*{\underline{Restricciones}}

\begin{itemize}[itemsep=0pt, parsep=0pt, leftmargin=1em]
    \item \underline{\textbf{icl$>$}}. Conecta una UW con su clase natural (hiperónimo: palabra cuyo significado engloba el de otros). \verb+bus(icl>vehículo)+. Algunos icl útiles:
    \begin{itemize}[itemsep=0pt, parsep=0pt, leftmargin=1em]
        \item \textbf{verb}. Forma más general de describir un verbo; evitar usar.
        \begin{itemize}[itemsep=0pt, parsep=0pt, leftmargin=1em]
            \item do. Acciones con un agente que las empieza.
            \item occur. Eventos sin un agente responsable.
            \item be. Situaciones estáticas (estados, propiedades, relaciones...)
        \end{itemize}
        \item \textbf{thing}. Forma más general de describir un sustantivo; evitar usar.
        \begin{itemize}[itemsep=0pt, parsep=0pt, leftmargin=1em]
            \item abstract\_thing (action, activity, process, quantity...)
            \item concrete\_thing (living\_thing(person, animal, plant), functional\_thing, matter, natural\_object)
            \item place.
        \end{itemize}
        \item \textbf{adj}. Forma más general de describir un adjetivo.
        \item \textbf{adv}. Forma más general de describir un adverbio.
    \end{itemize}
    \item \underline{\textbf{iof$>$}}. Conecta objetos únicos con el concepto del que son instancia. \verb+Barcelona(iof>city)+, \verb+Barcelona(iof>sport_club)+, \verb+Cervantes(iof>person)+.
    \item \textbf{equ$>$}. Indica un sinónimo de la UW. \verb+UN(equ>United_Nations)+
    \item \textbf{ant$>$}. Indica un antónimo de la UW. \verb+poor(ant>rich)+
    \item \textbf{pof$>$}. Indica que la UW es un componente de un todo. \verb+roof(pof>house)+.
    \item \textbf{com$>$}. Indica un componente importante del sentido de la UW. \verb+corner(icl>place, com>interior)+
\end{itemize}

\subsubsection*{\underline{Relaciones semánticas}}

\underline{Relaciones argumentales}:

\begin{itemize}[itemsep=0pt, parsep=0pt, leftmargin=1em]
    \item \textbf{agt() ó agt$>$}. Participante que inicia (con o sin voluntad)
    una acción (verbos bajo la categoría icl>do). \verb+marry(agt>human)+ si traducimos del español, pero \verb+marry(agt>male)+ o \verb+marry(agt>female)+ desde el ruso. \verb+agt(read(icl>do,agt>person,...),Alejandro(...))+.
    \item \textbf{obj() ó obj$>$}. Define el participante directamente afectado por un evento o estado. Incluye:
    \begin{itemize}[itemsep=0pt, parsep=0pt, leftmargin=1em]
        \item Sujeto de verbos que indican procesos (icl$>$occur). \underline{La ciudad} se inundó tras las fuertes lluvias.
        \item Objeto de verbos que indican actividades o acciones (icl$>$do). Diseñaron un nuevo \underline{plan}.
        \item Objeto de verbos que indican estados (icl$>$be). Tengo un \underline{bolígrafo}.
    \end{itemize}
    \item \textbf{cob() ó cob$>$}. Existe en paralelo con el objeto (obj) y está afectado por el evento. obj está en el primer plano mientras que cob está en el segundo plano. El AVE va a conectar Madrid [obj] con Málaga [cob].
    \item \textbf{aoj() ó aoj$>$}. Define el participante que se encuentra en un cierto estado o tiene cierta propiedad. La velocidad del \underline{avión}. \underline{María} quiere a Juan.
    \item \textbf{rec() ó rec$>$}. Recipiente. Doy una carta a \underline{María}.
    \item \textbf{gol() ó gol$>$}. Define el estado o propiedad que el objeto adquiere en la situación. Las luces cambiaron de rojo \underline{a verde}.
    \item \textbf{src() ó src$>$}. Define el estado inicial de un objeto. Las luces cambiaron de \underline{rojo} a verde.
\end{itemize}

\underline{Relaciones causales}:

\begin{itemize}[itemsep=0pt, parsep=0pt, leftmargin=1em]
    \item \textbf{rsn() ó rsn$>$}. Define la razón por la que ocurre un evento. No fueron a la montaña por el \underline{mal tiempo}.
    \item \textbf{pur() ó pur$>$}. Define el propósito u objetivo del agente de un evento. Trabaja para \underline{ganar dinero}.
\end{itemize}

\underline{Relaciones circunstanciales}:

\begin{itemize}[itemsep=0pt, parsep=0pt, leftmargin=1em]
    \item \textbf{met() ó met$>$}. Define los medios o el método por el cual se realiza una acción. Lo resolvió con \underline{un algoritmo}.
    \item \textbf{ins() ó ins$>$}. Define el instrumento con el que se ejecuta una acción. Un
    instrumento es siempre un objeto concreto (icl$>$concrete\_thing). Abrió la puerta con \underline{una ganzúa}.
    \item \textbf{man() ó man$>$}. Define la manera en que se realiza un evento o las características de un evento. Es una ciudad \underline{muy} bella.
\end{itemize}
    
\underline{Relaciones espaciales}:

\begin{itemize}[itemsep=0pt, parsep=0pt, leftmargin=1em]
    \item \textbf{plc() ó plc$>$}. Lugar. Se sentó \underline{debajo} del árbol.
    \item \textbf{plf() ó plf$>$}. Lugar de dónde. Vinieron a Madrid desde \underline{Granada}.
    \item \textbf{plt() ó plt$>$}. Lugar a dónde. Vinieron a \underline{Madrid} desde Granada.
    \item \textbf{snc() ó snc$>$}. Lugar virtual. Hay dificultades en \underline{este documento}.
\end{itemize}

\underline{Relaciones temporales}:

\begin{itemize}[itemsep=0pt, parsep=0pt, leftmargin=1em]
    \item \textbf{tim() ó tim$>$}. Tiempo. Llegaron todos a las \underline{cinco} de la tarde.
    \item \textbf{tmf() ó tmf$>$}. Tiempo inicial. Trabajó desde las \underline{ocho} de la mañana.
    \item \textbf{tmt() ó tmt$>$}. Tiempo final. Trabajó hasta las \underline{ocho} de la noche.
    \item \textbf{dur() ó dur$>$}. Duración. Durante su \underline{ausencia} las cosas fueron más fáciles.
\end{itemize}

\underline{Relaciones nominales}:

\begin{itemize}[itemsep=0pt, parsep=0pt, leftmargin=1em]
    \item \textbf{mod() ó mod$>$}. Una cosa que modifica a otra. Árbol \underline{verde}.
    \item \textbf{pos() ó pos$>$}. Poseedor de una cosa. Perro de \underline{Juan}.
    \item \textbf{pof() ó pof$>$}. Relación de parte de un todo. Ala del \underline{pájaro}. 
    \item \textbf{nam() ó nam$>$}. Nombre de algo. El río \underline{Po}. 
    \item \textbf{cnt() ó cnt$>$}. Forma distinta de denotar el objeto. Paquito, \underline{Jefazo}.
\end{itemize}

\underline{Relaciones lógicas}:

\begin{itemize}[itemsep=0pt, parsep=0pt, leftmargin=1em]
    \item \textbf{and() ó and$>$}. Conjunción y. Pedro y Pablo.
    \item \textbf{or() ó or$>$}. Disyunción o. Jueves o Viernes.
    \item \textbf{con() ó con$>$}. Condición a la realización del evento. Si \underline{vienes}, yo también me voy.
    \item \textbf{qua() ó qua$>$}. Cantidad. \underline{Cinco} lobitos.
    \item \textbf{bas() ó bas$>$}. Indica comparación. Joaquín es más listo que el \underline{resto}.
\end{itemize}

\end{multicols*}

\end{document}