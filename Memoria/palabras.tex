%----------------------------------------------------------------------------------------
%	PACKAGES AND OTHER DOCUMENT CONFIGURATIONS
%----------------------------------------------------------------------------------------

\documentclass{article}
\usepackage[spanish]{babel}
\usepackage[utf8]{inputenc}
\usepackage{titling}
\usepackage{mathtools}
\usepackage{float}
\usepackage{enumitem}
\usepackage{csquotes}
\usepackage{caption}
\usepackage{subcaption}
\usepackage{svg}
\usepackage{amsmath}
\usepackage[T1]{fontenc}
\usepackage{listings}
\usepackage{xparse}
\usepackage{xcolor}
\usepackage[most]{tcolorbox}
\usepackage[style=authoryear,giveninits=true, uniquename=init, maxcitenames=2, natbib]{biblatex}
\usepackage[citecolor=blue]{hyperref}

%----------------------------------------------------------------------------------------
%	ASSIGNMENT INFORMATION
%----------------------------------------------------------------------------------------

\newcommand\course{\emph{Máster Universitario en Inteligencia Artificial \\ Ingeniería Lingüística}}

\title{Práctica 1: Traducción a UNL} % Title of the assignment

\author{
        \normalsize Joaquín Jiménez López de Castro --- \small\texttt{jo.jimenez@alumnos.upm.es}\\
        \normalsize Ángel Fragua Baeza --- \small\texttt{angel.fragua@alumnos.upm.es}
}

%----------------------------------------------------------------------------------------

\input{structure.tex} % Include the file specifying the document structure and custom commands
\renewcommand{\lstlistingname}{Código}
\DeclarePairedDelimiter\ceil{\lceil}{\rceil}
\DeclarePairedDelimiter\floor{\lfloor}{\rfloor}
\DeclareNameAlias{sortname}{family-given}
\addbibresource{main.bib}
\DeclareFieldFormat[article,inbook,incollection,inproceedings,patent,thesis,unpublished]{citetitle}{#1\isdot}
\DeclareFieldFormat[article,inbook,incollection,inproceedings,patent,thesis,unpublished]{title}{#1\isdot}

\newenvironment{absolutelynopagebreak}
  {\par\nobreak\vfil\penalty0\vfilneg
   \vtop\bgroup}
  {\par\xdef\tpd{\the\prevdepth}\egroup
   \prevdepth=\tpd}


\tcbset{
frame code={}
center title,
left=0pt,
right=0pt,
top=0pt,
bottom=0pt,
colback=yellow!10,
colframe=white,
width=\dimexpr\textwidth\relax,
enlarge left by=0mm,
boxsep=5pt,
arc=0pt,outer arc=0pt,
}

\lstset{literate=
{á}{{\'a}}1 {é}{{\'e}}1 {í}{{\'i}}1 {ó}{{\'o}}1 {ú}{{\'u}}1
{Á}{{\'A}}1 {É}{{\'E}}1 {Í}{{\'I}}1 {Ó}{{\'O}}1 {Ú}{{\'U}}1
{à}{{\`a}}1 {è}{{\`e}}1 {ì}{{\`i}}1 {ò}{{\`o}}1 {ù}{{\`u}}1
{À}{{\`A}}1 {È}{{\'E}}1 {Ì}{{\`I}}1 {Ò}{{\`O}}1 {Ù}{{\`U}}1
{ä}{{\"a}}1 {ë}{{\"e}}1 {ï}{{\"i}}1 {ö}{{\"o}}1 {ü}{{\"u}}1
{Ä}{{\"A}}1 {Ë}{{\"E}}1 {Ï}{{\"I}}1 {Ö}{{\"O}}1 {Ü}{{\"U}}1
{â}{{\^a}}1 {ê}{{\^e}}1 {î}{{\^i}}1 {ô}{{\^o}}1 {û}{{\^u}}1
{Â}{{\^A}}1 {Ê}{{\^E}}1 {Î}{{\^I}}1 {Ô}{{\^O}}1 {Û}{{\^U}}1
{œ}{{\oe}}1 {Œ}{{\OE}}1 {æ}{{\ae}}1 {Æ}{{\AE}}1 {ß}{{\ss}}1
{ç}{{\c c}}1 {Ç}{{\c C}}1 {ø}{{\o}}1 {å}{{\r a}}1 {Å}{{\r A}}1
{€}{{\EUR}}1 {£}{{\pounds}}1
}

\NewDocumentCommand{\uw}{mmm}{
    \begin{absolutelynopagebreak}
        \item \underline{#1}
        \begin{itemize}[label={}]
            \item \texttt{#2}
            \def\temp{#3}\ifx\temp\empty
            \else
            \item \textbf{Ejemplo}: #3
            \fi
        \end{itemize}
    \end{absolutelynopagebreak}
}

\NewDocumentCommand{\uwlist}{mm}{
    \subsection*{#1}
    \begin{itemize}[itemindent=2em, label={$\blacktriangleright $}]
        #2
    \end{itemize}
}

\NewDocumentCommand{\unlexp}{mm}{
    \begin{figure}[H]
        \centering
        \begin{subfigure}[b]{0.9\textwidth}
            \begin{tcolorbox}\texttt{#1}\end{tcolorbox}
            \caption{Expresión UNL Oración #2}
            \label{unl:UNL_#2}
        \end{subfigure}
        \hfill \\
        \begin{subfigure}[b]{0.9\textwidth}
            \centering
            \includegraphics[width=\textwidth, height=10cm,keepaspectratio]{images/unl_#2.gv.png}
            \caption{Hipergrafo UNL Oración #2}
            \label{fig:UNL_#2}
        \end{subfigure}
    \end{figure}
}

\begin{document}

\maketitle

\thispagestyle{empty}

\newpage

\pagenumbering{arabic}

\section{Texto original}

En primer lugar, se ha escogido un texto, en este caso, es un fragmento del texto de \url{https://lingua.com/es/espanol/lectura/mi-habitacion/}, que cuenta con 199 palabras. El propósito original del texto es examinar a estudiantes del idioma español en el nivel básico-intermedio, y se ha escogido para evitar construcciones demasiado complicadas del idioma en esta primera toma de contacto con UNL. A continuación, se encuentra el fragmento de dicho texto, con los distintos tipos de palabras verbos, sustantivos, adjetivos y adverbios distinguidos mediante el código de colores verde, azul, rojo y naranja respectivamente.

\textcolor{teal}{Verbos}

\textcolor{blue}{Sustantivos}

\textcolor{red}{Adjetivos}

\textcolor{orange}{Adverbios}

Uno de mis \textcolor{blue}{lugares} \textcolor{red}{favoritos} de mi \textcolor{blue}{hogar} \textcolor{teal}{es} mi \textcolor{blue}{habitación}. \textcolor{orange}{Siempre} que \textcolor{teal}{llego} a \textcolor{blue}{casa} \textcolor{teal}{voy} \textcolor{orange}{directamente} a mi \textcolor{blue}{habitación} para \textcolor{teal}{descansar} y \textcolor{teal}{divertirme}. \textcolor{orange}{Allí} \textcolor{teal}{paso} la \textcolor{red}{mayor} \textcolor{blue}{parte} de mi \textcolor{blue}{tiempo} \textcolor{blue}{libre}, \textcolor{teal}{escuchando} \textcolor{blue}{música}, \textcolor{teal}{leyendo} \textcolor{blue}{libros} \textcolor{red}{interesantes} o \textcolor{teal}{dibujando} \textcolor{red}{todo} lo que \textcolor{teal}{puedo imaginar}.

Me \textcolor{teal}{siento} \textcolor{orange}{muy} \textcolor{red}{cómodo} en mi \textcolor{blue}{habitación}. \textcolor{orange}{Nada más} \textcolor{teal}{entrar} por la \textcolor{blue}{puerta} \textcolor{teal}{se ve} al \textcolor{blue}{fondo} la \textcolor{blue}{ventana} que \textcolor{teal}{deja entrar} una \textcolor{red}{agradable} \textcolor{blue}{luz}. Las \textcolor{blue}{paredes} \textcolor{teal}{están} \textcolor{teal}{pintadas} en un \textcolor{blue}{tono} \textcolor{red}{claro} y la \textcolor{blue}{lámpara} del \textcolor{blue}{techo} y la \textcolor{blue}{alfombra} \textcolor{teal}{tienen} el \textcolor{red}{mismo} \textcolor{blue}{color}.

La \textcolor{blue}{cama} \textcolor{teal}{es} \textcolor{orange}{muy} \textcolor{red}{cómoda} y \textcolor{teal}{es} \textcolor{orange}{donde} \textcolor{teal}{he colocado} unos \textcolor{blue}{juguetes} que me \textcolor{teal}{recuerdan} a mi \textcolor{blue}{infancia}. \textcolor{orange}{Encima} de la \textcolor{blue}{cama} \textcolor{teal}{es} el \textcolor{red}{mejor} \textcolor{blue}{lugar} para \textcolor{teal}{leer} mis \textcolor{blue}{novelas} \textcolor{red}{preferidas}. Me \textcolor{teal}{gusta} \textcolor{orange}{mucho} \textcolor{teal}{estirarme} \textcolor{orange}{encima} de la \textcolor{blue}{cama} y \textcolor{teal}{contemplar} los \textcolor{blue}{cuadros} de las \textcolor{blue}{paredes} de mi \textcolor{blue}{habitación}.

En una \textcolor{blue}{esquina} de la \textcolor{blue}{habitación} \textcolor{teal}{he colocado} un \textcolor{blue}{escritorio} con una \textcolor{blue}{silla} que \textcolor{teal}{es} el \textcolor{blue}{lugar} \textcolor{orange}{donde} \textcolor{teal}{estudio} o \textcolor{orange}{donde} \textcolor{teal}{utilizo} mi \textcolor{blue}{libro} de \textcolor{blue}{ejercicios} para \textcolor{teal}{aprender} \textcolor{orange}{mejor} el \textcolor{blue}{español}. El \textcolor{blue}{escritorio} \textcolor{teal}{es} \textcolor{red}{grande}, \textcolor{teal}{tengo} \textcolor{blue}{espacio} para \textcolor{red}{todas} mis \textcolor{blue}{cosas}. \textcolor{orange}{Encima} del \textcolor{blue}{escritorio} \textcolor{teal}{tengo} un \textcolor{blue}{estante} \textcolor{orange}{donde} \textcolor{teal}{almaceno} \textcolor{red}{todos} mis \textcolor{blue}{libros} y \textcolor{orange}{también} otros \textcolor{blue}{objetos} que me \textcolor{teal}{gustan}.

\section{Creación de UWs}

Utilizando un orden por tipo de palabra, y después el orden alfabético, se ha listado para cada palabra del texto su respectiva UW. Para las UW que necesitan argumentos, se ha incluido un ejemplo justo debajo, marcando con corchetes las palabras correspondientes a cada argumento. En algunos casos, distintas palabras del texto se han agrupado en una sola UW, por ejemplo, cuando solamente variaban en género o número. 

\uwlist{Listado de Sustantivos:}
{
    \uw
    {Alfombra}
    {carpet(icl>floor\_cover)}
    {}
    \uw
    {Cama}
    {bed(icl>bedroom\_furniture)}
    {}
    \uw
    {Casa}
    {house(icl>home)}
    {}
    \uw
    {Color}
    {colour(icl>visual\_property, aoj>concrete\_thing, obj>colour)}
    {El libro[aoj] es de color rojo[obj]}
    \uw
    {Cosas}
    {thing(icl>physical\_entity)}
    {}
    \uw
    {Cuadros}
    {painting(icl>graphic\_art)}
    {}
    \uw
    {Ejercicios}
    {exercise(icl>lesson, agt>human, pur>abstract\_thing)}
    {Yo[agt] utilizo mi libro de ejercicios de español [pur].}
    \uw
    {Escritorio}
    {desk(icl>table)}
    {}
    \uw
    {Espacio}
    {space(icl>location)}
    {}
    \uw
    {Español}
    {spanish(iof>romance\_language)}
    {}
    \uw
    {Esquina}
    {corner(icl>area, com>interior)}
    {}
    \uw
    {Estante}
    {shelf(icl>support)}
    {}
    \uw
    {Fondo}
    {background(icl>view, obj>concrete\_thing)}
    {Se ve al fondo de la habitación[obj] la ventana.}
    \uw
    {Habitación}
    {bedroom(icl>room)}
    {}
    \uw
    {Hogar}
    {home(icl>residence)}
    {}
    \uw
    {Infancia}
    {childhood(icl>time\_of\_life)}
    {}
    \uw
    {Juguetes}
    {toy(icl>artifact)}
    {}
    \uw
    {Libro, Libros}
    {book(icl>publication, agt>human, obj>abstract\_thing)}
    {Leo un libro de Cervantes[agt] sobre el español (obj)}
    \uw
    {Lugar}
    {place(icl>point, plc>place, obj>action)}
    {La cama[plc] es un lugar para dormir[obj]}
    \uw
    {Lugares}
    {place(icl>point, plc>place)}
    {El baño es uno de mis lugares favoritos de la casa[plc].}
    \uw
    {Luz}
    {light(icl>ilumination)}
    {}
    \uw
    {Lámpara}
    {lamp(icl>source\_of\_illumination)}
    {}
    \uw
    {Música}
    {music(icl>sound)}
    {}
    \uw
    {Novelas}
    {novel(icl>book, agt>human, obj>abstract\_thing)}
    {Leo una novela de Cervantes[agt] de caballerías[obj].}
    \uw
    {Objetos}
    {object(icl>physical\_entity)}
    {}
    \uw
    {Paredes}
    {wall(icl>divider, pof>room)}
    {}
    \uw
    {Parte}
    {part(icl>relation, aoj>thing, obj>thing)}
    {La habitación[aoj] es parte de la casa[obj]}
    \uw
    {Puerta}
    {door(icl>movable\_barrier)}
    {}
    \uw
    {Silla}
    {chair(icl>seat)}
    {}
    \uw
    {Techo}
    {roof(icl>protective\_covering)}
    {}
    \uw
    {Tiempo libre}
    {free\_time(icl>time\_period)}
    {}
    \uw
    {Tono}
    {tone(icl>colour, aoj>concrete\_thing, obj>abstract\_thing)}
    {El libro[aoj] es de un tono oscuro[obj] }
    \uw
    {Ventana}
    {window(icl>frame)}
    {}
}

\uwlist{Listado de Verbos:}
{
    \uw
    {Almaceno}
    {store(icl>do, agt>animal, obj>concrete\_thing, plc>place)}
    {Yo[agt] almaceno cocacola[obj] en la despensa[plc].}
    \uw
    {Aprender}
    {learn(icl>do, agt>animal, obj>abstract\_thing)}
    {Yo[agt] quiero aprender UNL[obj].}
    \uw
    {Contemplar}
    {contemplate(icl>do, agt>human, obj>concrete\_thing)}
    {Manuel[agt] contempla el paisaje[obj].}
    \uw
    {Descansar}
    {rest(icl>do, agt>animal)}
    {El panda[agt] descansa.}
    \uw
    {Dibujando}
    {draw(icl>do, agt>human, obj>thing)}
    {Yo[agt] estoy dibujando un perro[obj].}
    \uw
    {Divertirme}
    {have\_fun(icl>do, agt>human)}
    {Yo[agt] quiero divertirme.}
    \uw
    {Deja entrar}
    {let\_in(icl>be, aoj>thing, obj>thing, plt>thing)}
    {La ventana[aoj] deja entrar la luz[obj] a la habitación[plt].}
    \uw
    {Entrar}
    {enter(icl>do, agt>animal, plt>place, plf>place)}
    {Juan[agt] entra a su casa[plt] por la puerta trasera [plf]}
    \uw
    {Es, Están}
    {be(icl>be, obj>uw, aoj>uw)}
    {El coche[aoj] es azul[obj].\\
    Los niños[aoj] están felices[obj].}
    \uw
    {Escuchando}
    {listen(icl>do, agt>animal, obj>thing)}
    {Javier[agt] está escuchando música[obj].}
    \uw
    {Estirarme}
    {stretch(icl>do, agt>animal)}
    {Yo[agt] voy a estirarme.}
    \uw
    {Estudio}
    {study(icl>do, agt>human, obj>thing)}
    {Yo[agt] estudio las Redes Bayesianas[obj].}
    \uw
    {Gusta, Gustan}
    {like(icl>be, aoj>animal, obj>thing)}
    {A María[aoj] le gustan las galletas[obj].}
    \uw
    {He colocado}
    {place(icl>do, agt>animal, obj>concrete\_thing, plc>place)}
    {Yo[agt] he colocado mi ratón[obj] en el escritorio[plc].}
    \uw
    {Leer, Leyendo}
    {read(icl>do, agt>human, obj>information)}
    {Yo[agt] estoy leyendo un libro[obj].}
    \uw
    {Llego}
    {arrive(icl>do, agt>animal, plf>place, plt>place)}
    {Yo[agt] llego a casa[plt] desde el trabajo[plf].}
    \uw
    {Paso}
    {spend(icl>do, agt>concrete\_thing, dur>abstract\_thing, obj>process)}
    {El ordenador[agt] pasa 5 minutos[dur] computando[obj] cada día.}
    \uw
    {Pintadas}
    {paint(icl>do, agt>human, obj>thing, man>colour)}
    {El niño[agt] pinta una pared [obj] de rojo[man].}
    \uw
    {Puedo imaginar}
    {imagine(icl>do, obj>thing, agt>human)}
    {Juan[agt] imagina una historia[obj].}
    \uw
    {Recuerdan}
    {remind(icl>be, obj>thing, aoj>human, cob>thing)}
    {Las montañas[obj] me[aoj] recuerdan a mi infancia[cob].}
    \uw
    {Se ve}
    {see(icl>be, aoj>animal, obj>concret\_thing, plc>thing)}
    {La casa[obj] se ve por alguien[aoj] en la colina[plc].}
    \uw
    {Siento}
    {feel(icl>be, aoj>human, obj>abstract\_thing)}
    {Yo [aoj] me siento cómodo[obj] en mi habitación.}
    \uw
    {Tengo, Tienen}
    {have(icl>be, aoj>thing, obj>thing)}
    {Yo[aoj] tengo hambre[obj].\\
    Las camas[aoj] tienen muchos años[obj].}
    \uw
    {Utilizo}
    {use(icl>do, agt>animal, ins>thing, pur>uw)}
    {Yo[agt] utilizo la cama[ins] para dormir[pur].}
    \uw
    {Voy}
    {go(icl>do, agt>animal, plf>place, plt>place)}
    {Yo[agt] voy de mi casa[plf] al campo[plt]}
}

\uwlist{Listado de Adverbios:}
{
    \uw
    {Allí}
    {there(icl>adv)}
    {}
    \uw
    {Directamente}
    {directly(icl>adv)}
    {}
    \uw
    {Donde}
    {where(icl>adv)}
    {}
    \uw
    {Encima}
    {above(icl>adv, obj>concrete\_thing)}
    {El plato esta encima de la mesa[obj].}
    \uw
    {Mejor}
    {better(icl>adv)}
    {}
    \uw
    {Mucho}
    {a\_lot(icl>adv)}
    {}
    \uw
    {Muy}
    {very(icl>adv)}
    {}
    \uw
    {Nada más}
    {as\_soon\_as(icl>adv)}
    {}
    \uw
    {Siempre}
    {always(icl>adv)}
    {}
    \uw
    {También}
    {also(icl>adv)}
    {}
}

\uwlist{Listado de Adjetivos:}
{
    \uw
    {Agradable}
    {pleasant(icl>adj)}
    {}
    \uw
    {Claro}
    {light(icl>adj)}
    {}
    \uw
    {Cómodo}
    {comfortable(icl>adj)}
    {}
    \uw
    {Favoritos, Preferidas}
    {favorite(icl>adj)}
    {}
    \uw
    {Grande}
    {big(icl>adj)}
    {}
    \uw
    {Interesantes}
    {interesting(icl>adj)}
    {}
    \uw
    {Mayor}
    {most(icl>adj)}
    {}
    \uw
    {Mejor}
    {best(icl>adj)}
    {}
    \uw
    {Mismo}
    {same(icl>adj, aoj>uw, obj>uw)}
    {Eso[aoj] es del mismo color que lo otro[obj]}
    \uw
    {Todo, Todos, todas}
    {all(icl>adj)}
    {}
}

\section{Expresiones UNL}

A continuación, se ha creado una expresión UNL para las distintas oraciones del texto. Cada una contiene la oración original del texto, y su correspondiente traduucción a UNL usando las UW anteriormente creadas. Además, usando un programa en Python (véase Anexo A), se ha creado una visualización del grafo correspondiente a cada una.

\obeylines\unlexp{
    [S:1]
    \{org\}
        Uno de mis lugares favoritos de mi hogar es mi habitación.
    \{/org\}
    \{unl\}
        pof(bedroom.@sg.@def, home.@sg.@def)
        aoj(be.@present, bedroom.@sg.@def)
        obj(be.@present, place.@entry.@pl.@indef)
        mod(place.@entry.@pl.@indef, favorite)
    \{/unl\}
    [/S]
}{1}

\obeylines\unlexp{
    [S:2]
    \{org\}
        Siempre que llego a casa voy directamente a mi habitación para descansar y divertirme.
    \{/org\}
    \{unl\}
        man(go.@entry.@present, directly)
        plt(go.@entry.@present, bedroom.@sg.@def)
        pur(go.@entry.@present, rest)
        and(rest, have\_fun)
        tim(go.@entry.@present, arrive.@present)
        man(arrive.@present, always)
        plt(arrive.@present, house.@sg.@def)
    \{/unl\}
    [/S]
}{2}

\obeylines\unlexp{
    [S:3]
    \{org\}
        Allí paso la mayor parte de mi tiempo libre, escuchando música, leyendo libros interesantes o dibujando todo lo que puedo imaginar.
    \{/org\}
    \{unl\}
        plc(spend.@entry.@present, there)
        dur(spend.@entry.@present, free\_time.@def)
        qua(free\_time.@def, part.@sg.@def)
        mod(part.@sg.@def, most)
        obj(spend.@entry.@present, listen.@progress)
        obj(listen.@progress, music.@indef)
        or(listen.@progress, read.@progress)
        obj(read.@progress, book.@pl.@indef)
        mod(book.@pl.@indef, interesting)
        or(read.@progress, draw.@progress)
        obj(draw.@progress, all)
        qua(all, imagine.@possibility)
    \{/unl\}
    [/S]
}{3}

\obeylines\unlexp{
    [S:4]
    \{org\}
        Me siento muy cómodo en mi habitación.
    \{/org\}
    \{unl\}
        obj(feel.@present, comfortable.@entry.@sg)
        man(comfortable.@entry.@sg, very)
        plc(comfortable.@entry.@sg, bedroom.@sg.@def)
    \{/unl\}
    [/S]
}{4}

\obeylines\unlexp{
    [S:5]
    \{org\}
        Nada más entrar por la puerta se ve al fondo la ventana que deja entrar una agradable luz.
    \{/org\}
    \{unl\}
        tim(see.@entry.@present, enter)
        man(enter, as\_soon\_as)
        plf(enter, door.@sg.@def)
        plc(see.@entry.@present, background.@def)
        obj(see.@entry.@present, window.@sg.@def)
        mod(window.@sg.@def, let\_in.@present)
        obj(let\_in.@present, light.@sg.@indef)
        mod(light.@sg.@indef, pleasant)
    \{/unl\}
    [/S]
}{5}

\obeylines\unlexp{
    [S:6]
    \{org\}
        Las paredes están pintadas en un tono claro y la lámpara del techo y la alfombra tienen el mismo color.
    \{/org\}
    \{unl\}
        and(be.@present, have.@present)
        obj(be.@present, paint.@entry)
        aoj(be.@present, wall.@pl.@def)
        man(paint.@entry, tone.@sg.@indef)
        mod(tone.@sg.@indef, light)
        obj(have.@present, same)
        bas(same, colour.@sg.@def)
        aoj(have.@present, lamp.@sg.@def)
        pof(lamp.@sg.@def, roof.@sg.@def)
        and(lamp.@sg.@def, carpet.@sg.@def)
    \{/unl\}
    [/S]
}{6}

\obeylines\unlexp{
    [S:7]
    \{org\}
        La cama es muy cómoda y es donde he colocado unos juguetes que me recuerdan a mi infancia.
    \{/org\}
    \{unl\}
        aoj(be.@present, bed.@sg.@def)
        obj(be.@present, comfortable.@entry)
        man(comfortable.@entry, very)
        and(comfortable.@entry, place.@past)
        plc(place.@past, where)
        obj(place.@past, toy.@pl.@indef)
        obj(remind.@present, toy.@pl.@indef)
        cob(remind.@present, childhood.@sg.@def)
    \{/unl\}
    [/S]
}{7}

\obeylines\unlexp{
    [S:8]
    \{org\}
        Encima de la cama es el mejor lugar para leer mis novelas preferidas.
    \{/org\}
    \{unl\}
        obj(be.@present, place.@sg.@def)
        aoj(be.@present, above)
        obj(above, bed.@sg.@def)
        man(place.@sg.@def, best)
        pur(place.@sg.@def, read)
        obj(read, novel.@pl.@def)
        mod(novel.@pl.@def, favorite)
    \{/unl\}
    [/S]
}{8}

\obeylines\unlexp{
    [S:9]
    \{org\}
        Me gusta mucho estirarme encima de la cama y contemplar los cuadros de las paredes de mi habitación.
    \{/org\}
    \{unl\}
        obj(like.@entry, stretch)
        plc(stretch, above)
        obj(above, bed.@sg.@def)
        man(like.@entry, very)
        and(stretch, contemplate)
        obj(contemplate, painting.@pl.@def)
        plc(painting.@pl.@def, wall.@pl.@def)
        pof(wall.@pl.@def, bedroom.@sg.@def)
    \{/unl\}
    [/S]
}{9}

\obeylines\unlexp{
    [S:10]
    \{org\}
        En una esquina de la habitación he colocado un escritorio con una silla que es el lugar donde estudio o donde utilizo mi libro de ejercicios para aprender mejor el español.
    \{/org\}
    \{unl\}
        plc(place.@entry.@past, corner.@sg.@indef)
        pof(corner.@sg.@indef, bedroom.@sg.@def)
        obj(place.@entry.@past, desk.@sg.@indef)
        and(desk.@sg.@indef, chair.@sg.@indef)
        plc(study.@present, desk.@sg.@indef)
        or(study.@present, use.@present)
        ins(use.@present, book.@sg.@def)
        obj(book.@sg.@def, exercise.@pl.@def)
        pur(exercise.@pl.@def, learn)
        man(learn, better)
        obj(learn, spanish.@sg.@def)
    \{/unl\}
    [/S]
}{10}

\obeylines\unlexp{
    [S:11]
    \{org\}
        El escritorio es grande, tengo espacio para todas mis cosas.
    \{/org\}
    \{unl\}
        obj(be.@present, big.@entry)
        aoj(be.@present, desk.@sg.@def)
        and(be.@present, have.@present)
        obj(have.@present, space.@sg.@indef)
        pur(space.@sg.@indef, thing.@pl.@def)
        mod(thing.@pl.@def, all)
    \{/unl\}
    [/S]
}{11}

\obeylines\unlexp{
    [S:12]
    \{org\}
        Encima del escritorio tengo un estante donde almaceno todos mis libros y también otros objetos que me gustan.
    \{/org\}
    \{unl\}
        plc(shelf.@sg.@indef, above)
        obj(above, desk.@sg.@def)
        obj(have.@entry.@present, shelf.@sg.@indef)
        plc(store.@present, shelf.@sg.@indef)
        obj(store.@present, book.@pl.@def)
        mod(book.@pl.@def, all)
        and(book.@pl.@def, object.@pl.@indef)
        obj(like.@present, object.@pl.@indef)
        man(like.@present, also)
    \{/unl\}
    [/S]
}{12}

\newpage
\subsection*{Anexo A. Programa Python para convertir UNL a imágenes}

\begin{lstlisting}[language=Python]
import string

# Librerias adicionales a instalar
import regex as re
from graphviz import Digraph
from cairosvg import svg2png

# Fichero de entrada, listado de expresiones UNL:
# [S:<número de oración>]
# {org}
# <oración original>
# {/org}
# {unl}
# <contenido>
# {/unl}
# [/S]
#
# [S:<número siguiente oración>]
# ...
input_file = 'grafos.txt'

# Directorio de salida
output_dir = '../Memoria/images'


def make_graph(edges, name):
    edges = clean_empty_edges(edges)
    nodes = extract_edges_nodes(edges)
    dot = Digraph(name=name)
    for node in nodes:
        dot.node(label=node, name=nodes[node])
    for e in edges:
        e_label = extract_edge_label(e)
        e_nodes = extract_edge_nodes(e)
        dot.edge(nodes[e_nodes[0]], nodes[e_nodes[1]], label=e_label)
    dot.render(directory=output_dir, format='svg')
    svg2png(url=f'{output_dir}{name}.gv.svg',
            write_to=f'{output_dir}{name}.gv.png',
            dpi=200)


def clean_empty_edges(edges):
    return list(filter(lambda e: bool(e), edges))


def extract_edge_nodes(edge):
    return re.findall(r'\(.*?\)', ''.join(edge.split()))[0][1:-1].split(',')


def extract_edges_nodes(edges):
    nodes = dict()
    ids = list(string.ascii_lowercase) + list(string.ascii_uppercase)
    id_idx = 0
    for e in edges:
        for n in extract_edge_nodes(e):
            nodes[n] = ids[id_idx]
            id_idx += 1
    return nodes


def extract_edge_label(edge):
    edge = ''.join(edge.split())
    return edge[:edge.index('(')]


with open(input_file, 'r') as f:
    text = ''.join(f.readlines())
    graphs = re.findall(r'\{unl\}(.*?)\{/unl\}', text, re.DOTALL)
    for i, g in enumerate(graphs):
        make_graph(g.splitlines(), f'unl_{i+1}')
\end{lstlisting}

\end{document}
